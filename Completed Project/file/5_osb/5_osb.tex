\documentclass[main.tex]{subfiles}
\begin{document}
\subsection{Observations}
We discovered that the TACACS+ protocol is relatively old (no updated RFC since around 1998), but still relevant today. Because of this, it would still be possible to execute the attacks that we have in this lab in a fairly large infrastructure with no means of mitigation. While writing the lab report and lab instructions, Exercise 1 and 3 (Isaac) contain machines of different subnets / networks as 2 and 4 (Troy). This is because we were working in two different environments to finalize the project. Overall, we have learned much information in researching this protocol and will continue to utilize our skills in future "real-world" scenarios.

\subsection{Suggestions}
If we had more time to work on this project, we would continue to search for ways to successfully implement the DOS attack on the TACACS+ server, as well as the mitigation on MITM attacks.

\subsection{Best Practices}
Best practices included:
\begin{itemize}
    \item Setting up a weekly meeting time to work on the project
    \item Organizing the lab exercises so that they flow nicely
    \item Contacting the creator of the TacoTaco project for assistance with Exercise 4 (Alexey Greendog Tyurin)
    \item Working together on the presentation and one-pager
    \item Emulating an environment in the lab on our laptops
    \item Creation of video demos
\end{itemize}

\end{document}