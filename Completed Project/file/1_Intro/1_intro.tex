\documentclass[main.tex]{subfiles}
\begin{document}
\label{intro}
When there are many network devices (routers, firewalls, switches) in an environment, it is useful to have a centralized management service for access to these devices. These services are known as Authentication, Authorization, and Accounting (AAA) services.

With TACACS+, these services are possible. A server running TACACS+ has the ability to set specific privileges & commands to particular users and user groups. The TACACS+ server then logs all aspects of the AAA communication (who is logged in, what commands they are issuing, etc.).

In this lab we make use of multiple VMs (Windows 7, Ubuntu Server / Desktop, and Kali Linux) running on either an Ubuntu Desktop or OSX host. In order to test the functionality of TACACS+ in a "real-world" network situation, we also make use of the GNS3 Software. We were able to connect the virtual machines in GNS3 with the switches it provides, as well as utilizing a Cisco 3600 Router image as the TACACS+ enabled device to connect to. This router image was provided to us by Dr. Salib. 

After we discovered how to implement the TACACS+ protocol in a "real-world" scenario, we began to assess the vulnerabilities that were inherent in the protocol. Some of these vulnerabilities included errors in the source code that could allow for a Denial-of-Service attack, and errors in the encryption method that TACACS+ uses for its packets that allows for a Man-in-the-Middle attack.
\hfill \break
All relevant attachments to this lab report can be found in this section: \ref{LOA}

\end{document}