%<*mytag1>
\begin{adjustwidth}{15pt}{0pt}
\textbf{Q1:}What  does  the  len  variable  represent?   (No  need  to  be  overlytechnical)
\\
\textbf{A1:}It is the length of the Tacacs+ header and the payload combined.
\end{adjustwidth}
%</mytag1>

%<*mytag2>
\begin{adjustwidth}{15pt}{0pt}
\textbf{Q2:} Why did we need the script to examine a Tacacs+ packet with scapy?  Hint:  Look online for a list of protocol layers that scapyknows by default.
\\
\textbf{A2:}The Tacacs+ layer is not included with scapy by default. The script creates a Tacacs+ layer that we can now interact with and send using scapy.
\end{adjustwidth}
%</mytag2>

%<*mytag3>
\begin{adjustwidth}{15pt}{0pt}
\textbf{Q3:}What is the script doing?  (Give a general answer)
\\
\textbf{A3:} It is simulating the TCP 3-Way-Handshake and sending a packet with a overly-large length field.
\end{adjustwidth}
%</mytag3>

%<*mytag4>
\begin{adjustwidth}{15pt}{0pt}
\textbf{Q4:}What does this firewall rule do and why do we need it?  (Hint try running the script without it.)
\\
\textbf{A4:}The kernel will send a TCP RST packet if it receives an SYN/ACK destined for a process running in user mode. This tells the firewall to drop the outgoing RST packets.
\end{adjustwidth}
%</mytag4>

%<*mytag5>
\begin{adjustwidth}{15pt}{0pt}
\textbf{Q5:}Why did the attack fail this time?
\\
\textbf{A5:}There was not a large enough payload for the attack to overload the server. This limiting factor is due to the implementation of scapy.
\end{adjustwidth}
%</mytag5>

%<*mytag6>
\begin{adjustwidth}{15pt}{0pt}
\textbf{Q6:} Confirm that krb5kdc and kadmin are up and running.
\\
\textbf{A6:}
\end{adjustwidth}
%</mytag6>

%<*mytag7>
\begin{adjustwidth}{15pt}{0pt}
\textbf{Q7:}Describe briefly what these files are for and how they are used by the Kerberos platform.
\\
\textbf{A7:}
\end{adjustwidth}
%</mytag7>

%<*mytag8>
\begin{adjustwidth}{15pt}{0pt}
\textbf{Q8:}Provide a copy of the principals already established, by default, on the Kerberos server.  Describe briefly what each represents.
\\
\textbf{A8:}
\end{adjustwidth}
%</mytag8>

%<*mytag9>
\begin{adjustwidth}{15pt}{0pt}
\textbf{Q9:}Describe in details the messages exchanged and captured on Wireshark (lo interface) for the locally initiated request for a TGT.  Make use of the klist response and the exported Kerberos packets. 
\\
\textbf{A9:}
\end{adjustwidth}
%</mytag9>

%<*mytag10>
\begin{adjustwidth}{15pt}{0pt}
\textbf{Q10:}Explain in details the Kerberos packets exchanged between VM1 and VM3 in the process of creating a TGT.  Hint: Make extensive use of the Wireshark captured packets!
\\
\textbf{A10:}
\end{adjustwidth}
%</mytag10>


%<*mytag11>
\begin{adjustwidth}{15pt}{0pt}
\textbf{Q11:}s there a way to peek into the key.tab file?
\\
\textbf{A11:}
\end{adjustwidth}
%</mytag11>


%<*mytag12>
\begin{adjustwidth}{15pt}{0pt}
\textbf{Q12:}From the debug messages generated (by selecting the –V option) explain the high level steps that led the SSH server to GSSAPI (Kerberos related) authentication over the password method.
\\
\textbf{A12:}
\end{adjustwidth}
%</mytag12>


%<*mytag13>
\begin{adjustwidth}{15pt}{0pt}
\textbf{Q13:}Correlate the packets you captured on VM1 with those captured on VM3.
\\
\textbf{A13:}
\end{adjustwidth}
%</mytag13>


%<*mytag14>
\begin{adjustwidth}{15pt}{0pt}
\textbf{Q14:}Compare the ssh debug messages in this case with the ones you explained in the answer to Q12.
\\
\textbf{A14:}
\end{adjustwidth}
%</mytag14>