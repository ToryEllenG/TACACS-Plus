\documentclass[main.tex]{subfiles}
\begin{document}
\subsection{Exercise 3: Denial-of-Service attack on TACACS+ with use of Scapy. (Isaac)}

\subsubsection{Analysis \& Evidence:}


\paragraph{Step 1: Setup and DOS}
\hfill \break

We begin by making sure that scapy is installed with \texttt{dpkg -s scapy}, and see that it is already installed.

\ExecuteMetaData[file/3_p1_ex3/3_p1_ex3_img]{mtag1}

Then, we create and edit the tacacs\_scapy.py file based on Appendix E.

\ExecuteMetaData[file/3_p1_ex3/3_p1_ex3_img]{mtag2}

Next, we make the script executable with \texttt{chmod 755 tacacs\_scapy.py} and run the script. Then, we run the scapy commands and get the following output:

\ExecuteMetaData[file/3_p1_ex3/3_p1_ex3_img]{mtag3}

\ExecuteMetaData[file/3_p1_ex3/3_p1_ex3_QAs]{mytag2}

\paragraph{Step 2: }
\hfill \break

We begin by setting up the Ubuntu Server VM with 512 MB of ram and restarting the VM. 

\ExecuteMetaData[file/3_p1_ex3/3_p1_ex3_img]{mtag4}

Next, we create the text file for the script in Appendix C and make it executable using \texttt{chmod}. 

\ExecuteMetaData[file/3_p1_ex3/3_p1_ex3_img]{mtag5}

Then we copy the script into the file.

\ExecuteMetaData[file/3_p1_ex3/3_p1_ex3_QAs]{mytag3}

Next we open Wireshark on the Ubuntu Server from the host, with the filter set to only display packets using port 49.

\ExecuteMetaData[file/3_p1_ex3/3_p1_ex3_img]{mtag6}

Next, we ran the iptables command to drop outgoing RST packets:

\ExecuteMetaData[file/3_p1_ex3/3_p1_ex3_img]{mtag7}
\ExecuteMetaData[file/3_p1_ex3/3_p1_ex3_QAs]{mytag4}

Then, we run the script, and check Wireshark.

\ExecuteMetaData[file/3_p1_ex3/3_p1_ex3_img]{mtag8}

Then we check on the server if Tacacs+ is still running.

\ExecuteMetaData[file/3_p1_ex3/3_p1_ex3_img]{mtag9}

Next, we examine the man page for malloc on the server.

\ExecuteMetaData[file/3_p1_ex3/3_p1_ex3_img]{mtag10}

Then, we copy the script in Appendix D and make it executable like the previous scripts.

\ExecuteMetaData[file/3_p1_ex3/3_p1_ex3_img]{mta11}

Next, we restarted tac\_plus and the wireshark capture and executed the attack.

\ExecuteMetaData[file/3_p1_ex3/3_p1_ex3_img]{mtag12}
\ExecuteMetaData[file/3_p1_ex3/3_p1_ex3_img]{mtag13}
\ExecuteMetaData[file/3_p1_ex3/3_p1_ex3_img]{mtag14}

\ExecuteMetaData[file/3_p1_ex3/3_p1_ex3_QAs]{mytag5}
    
\subsubsection{Key Learning/Takeaways: }
In this exercise we learned that when implementing an attack, it is very important to choose the correct tools. Scapy wasn't designed to send to send a packet with an outrageous payload. I imagine an attacker would have to write software from scratch to implement this attack successfully. I also believe it would be difficult to find a version of tacacs that hasn't been updated since the late 1990s. That being said, I'm sure this vulnerable version is running out there somewhere. 

\end{document}