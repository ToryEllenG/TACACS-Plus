\documentclass[main.tex]{subfiles}
\begin{document}
\subsection{Exercise 2 - TACACS on Ubuntu & Comparison to Windows }
\newlist{questions}{enumerate}{1}
\setlist[questions,1]{label={$\circ$ Q\arabic* -}}
\begin{itemize}

\subsubsection{Step 1: Preparation}

\begin{enumerate}[noitemsep,label=$\bullet$,leftmargin=20mm,labelsep=0.5cm]    
    
    \item For this exercise we will need a single Ubuntu Server VM, configured for NAT. We will refer to this machine as VM1
    
    \item On VM1 run \texttt{sudo su} to become the root user.
    
    \item This version of TACACS relies on four dependencies: gcc, bison, flex, and libwrap0-dev. Check to see if they are installed by running 
    \ExecuteMetaData[file/4_ex2/4_ex2_CBQMs]{mytag13}

\ExecuteMetaData[file/4_ex2/4_ex2_CBQMs]{mytag1}
 

     \item Use \texttt{apt-get install} to install the packages that aren't already installed.
     
\item Now we need to install TACACS+. Download by running 

\ExecuteMetaData[file/4_ex2/4_ex2_CBQMs]{mytag12}
     
     \item Navigate to the \texttt{tacacs+-F4.0.4.26} directory and use \texttt{cat} or \texttt{less} to view the contents of INSTALL.
     
\ExecuteMetaData[file/4_ex2/4_ex2_CBQMs]{mytag2}

     \item Now run 
     \ExecuteMetaData[file/4_ex2/4_ex2_CBQMs]{mytag14}to view all the installation options.
     
    \item Next, run 
     \ExecuteMetaData[file/4_ex2/4_ex2_CBQMs]{mytag15}
    
    \item Now, we must make sure that necessary library links are installed. Execute \texttt{nano /etc/ld.so.conf}. Add the line \texttt{/usr/local/lib}, and reload the libraries using the \texttt{ldconfig} command.
    
\end{enumerate}

    \subsubsection{Step 2:  }
    
\begin{enumerate}[noitemsep,label=$\bullet$,leftmargin=20mm,labelsep=0.5cm] 

    \item Now, we will be creating the two user groups: sys\textunderscore admin and network\textunderscore admin. Let's first create the tac\textunderscore plus.conf file.
    
    \item Execute \texttt{mkdir /etc/tacacs}, then change into that directory. Run \texttt{touch tac\textunderscore plus.conf} and \texttt{chmod 755 tac\textunderscore plus.conf}.
    
    \item Next, we shall make a file for the accounting logs. 
    \item \texttt{mkdir /var/log/tac\textunderscore plus}
    \item \texttt{touch /var/log/tac\textunderscore plus/tac\textunderscore plus.acct}
    
    \item Now, add the text in Appendix (appendix here) to the \texttt{tac\textunderscore plus.conf} file 
    
    \item Let's take a look at the file:
     
    \item Under \texttt{# Encryption Key} we see \texttt{key = "tac\textunderscore test}. This is used to encrypt packets between the server and clients.
    
    \item Now, look under the comment \texttt{# Set where to send accounting records}. We have the lines \texttt{accounting syslog;} and \texttt{accounting file = /var/log/tac\textunderscore plus/tac\textunderscore plus.acct}. This tells the tacacs daemon where to write log information.
    
    \item Let's take a look at the text under the \texttt{# ACL for network\textunderscore admin group} and the \texttt{# ACL for sys\textunderscore admin group} comments.
    
    \ExecuteMetaData[file/4_ex2/4_ex2_CBQMs]{mytag3}
    
    \item Now let's look at the sys\textunderscore admin group configuration, specifically the sections that begin with \texttt{cmd = }
    
    \ExecuteMetaData[file/4_ex2/4_ex2_CBQMs]{mytag4}
    
    \item Let's look at the at the specific users that are defined in our config file. These configurations can override those for the group specified.
    
    \item Under the user jonathanm, notice the \texttt{login = } and \texttt{enable = } fields. These are passwords generated with \texttt{tac\_pwd}. Instead of using the ones in the template let's make our own.
    
    \ExecuteMetaData[file/4_ex2/4_ex2_CBQMs]{mytag16}
    \ExecuteMetaData[file/4_ex2/4_ex2_CBQMs]{mytag17}
    
    \item The output you receive is a DES encrypted password. Add your password to the config file (the one you just generated and two more). 
    
    
\end{enumerate}


\subsubsection{Step 3: }

\begin{enumerate}[noitemsep,label=$\bullet$,leftmargin=20mm,labelsep=0.5cm]    

\item We need a couple more steps to give tac\_plus the normal start/stop abilities (using service)

\item Execute:

\ExecuteMetaData[file/4_ex2/4_ex2_CBQMs]{mytag18}
\ExecuteMetaData[file/4_ex2/4_ex2_CBQMs]{mytag19}
\ExecuteMetaData[file/4_ex2/4_ex2_CBQMs]{mytag20}

\item Now copy the script text in Appendix(blah) into \texttt{/etc/default/tac\textunderscore plus}

\item Start the service by executing:

\ExecuteMetaData[file/4_ex2/4_ex2_CBQMs]{mytag21}

\item Verify tac\textunderscore plus is in fact running.

\ExecuteMetaData[file/4_ex2/4_ex2_CBQMs]{mytag22}
\ExecuteMetaData[file/4_ex2/4_ex2_CBQMs]{mytag23}

\end{enumerate}
     
\end{itemize}
\end{document}